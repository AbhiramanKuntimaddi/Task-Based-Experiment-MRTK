\documentclass{article}

% Language setting
% Replace `english' with e.g. `spanish' to change the document language
\usepackage[english]{babel}

% Set page size and margins
% Replace `letterpaper' with `a4paper' for UK/EU standard size
\usepackage[a4paper,top=2cm,bottom=2cm,left=3cm,right=3cm,marginparwidth=1.75cm]{geometry}

% Useful packages
\usepackage{amsmath}
\usepackage{graphicx}
\usepackage[colorlinks=true, allcolors=blue]{hyperref}

\usepackage[T1]{fontenc}
\usepackage{tgbonum}

\usepackage{enumitem}

\usepackage{geometry}
\usepackage{array}

\renewcommand{\thesection}{\Roman{section}}
\renewcommand{\thesubsection}{\thesection.\Roman{subsection}}
\renewcommand{\thesubsubsection}{\thesection.\alph{subsubsection}}

\newenvironment{subs}
  {\adjustwidth{3em}{0pt}}
  {\endadjustwidth}

\title{\textbf{Task Sheet}}
\author{Abhiraman Kuntimaddi}
\date{February 25, 2022}

\begin{document}
\maketitle

\section*{Introduction to Experiment}
The aim of the project is integration of XR-technologies into the ECM process.
Due to the amalgamation of a number of platforms / devices, operating systems and APIs,
the use of a Cross Platform Development Framework (CPDF) is highly desirable. Deciding over
which CPDF platform turns out to be the best choice in the concrete application case is also the subject of this research project.\\~\\
In order to choose the best implementation choice, two scenarios are developed in Unity 3D Game engine using both MRTK(Native) and MRTK OpenXR(CPDF).
Two designed scenarios are as follows:
\begin{enumerate}
	\item \textbf{Scenario 1} : An XR Game(Whack-A-Mole)
	\item \textbf{Scenario 2}: XR Playground with objects in them to perform some simple tasks

\end{enumerate}

Time data and other relative data is noted down for the mentioned tasks below.

\section*{Tasks for \textbf{Scenario 1}}
\begin{table}[htb]
	\centering
	\setlength{\leftmargini}{0.5cm}
	\begin{tabular}{| m{0.75cm} | m{10cm} |}
		\hline
		\textbf{Task ID} & \textbf{Task Data}                                                                                                                                       \\
		\hline \hfill \break
		\textbf{1}       &
		\begin{itemize}[label={}]
			\item \textbf{Start the game by clicking on the "Start the Game"}
			\item \textit{Was the task performed as intended :}
		\end{itemize}                                     \\
		\hline \hfill \break
		\textbf{2}       &
		\begin{itemize}[label={}]
			\item \textbf{Pick up the hammer that's in the scene and wait till the countdown ends.}
			\item \textit{Was the task performed as intended :}
		\end{itemize} \\
		\hline \hfill \break
		\textbf{3}       &
		\begin{itemize}[label={}]
			\item \textbf{Once the game starts, Use the hammer to hit the moles and try to make a high score before the time runs out.}
			\item \textit{Total number of moles hit(Total Score):}
			\item \textit{Was the task performed as intended :}
		\end{itemize} \\
		\hline
	\end{tabular}
\end{table}

\newpage
\section*{Tasks for \textbf{Scenario 2}}
\begin{table}[htb]
	\centering
	\setlength{\leftmargini}{0.5cm}
	\begin{tabular}{| m{0.75cm} | m{10cm} |}
		\hline
		\textbf{Task ID} & \textbf{Task Data}                                                                                                                                       \\
		\hline \hfill \break
		\textbf{1}       &
		\begin{itemize}[label={}]
			\item \textbf{Use the joystick on the Right Motion Controllers to teleport and use the same right Motion Controller to point on UI elements and use the trigger to select the options.}
			\item \textit{Time Taken(in Secs) to do the above task :}
			\item \textit{Was the task performed as intended :}
		\end{itemize}                                     \\
		\hline \hfill \break
		\textbf{2}       &
		\begin{itemize}[label={}]
			\item \textbf{Use the Motion Controllers to move the cubes from the ground to the green plate in the room and proceed to the next room.}
			\item \textit{Time Taken(in Secs) to do the above task :}
			\item \textit{Was the task performed as intended :}
		\end{itemize}                                     \\
		\hline
		\textbf{3}       &
		\begin{itemize}[label={}]
			\item \textbf{Use the Motion Controllers to scale the colored 3D-Objects on the ground and place them on their respective colored plates and then proceed to the next room.}
			\item \textit{Time Taken(in Secs) to do the above task :}
			\item \textit{Was the task performed as intended :}
		\end{itemize} \\
		\hline \hfill \break
		\textbf{4}       &
		\begin{itemize}[label={}]
			\item \textbf{Delete the 3D-Objects that are colored in red using the button on top of the 3-D Object.}
			\item \textit{Time Taken(in Secs) to do the above task :}
			\item \textit{Was the task performed as intended :}
		\end{itemize} \\
		\hline \hfill \break
		\textbf{5}       &
		\begin{itemize}[label={}]
			\item \textbf{Use the Menu in the scene to add new 3-D Objects to the scene and follow the instructions that are written on tooltips to replace the Deleted objects with new ones.}
			\item \textit{Time Taken(in Secs) to do the above task :}
			\item \textit{Was the task performed as intended :}
		\end{itemize} \\
		\hline
	\end{tabular}
\end{table}


\end{document}